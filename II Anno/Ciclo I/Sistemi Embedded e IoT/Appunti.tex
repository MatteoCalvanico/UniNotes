% Options for packages loaded elsewhere
\PassOptionsToPackage{unicode}{hyperref}
\PassOptionsToPackage{hyphens}{url}
\documentclass[
]{article}
\usepackage{xcolor}
\usepackage{amsmath,amssymb}
\setcounter{secnumdepth}{-\maxdimen} % remove section numbering
\usepackage{iftex}
\ifPDFTeX
  \usepackage[T1]{fontenc}
  \usepackage[utf8]{inputenc}
  \usepackage{textcomp} % provide euro and other symbols
\else % if luatex or xetex
  \usepackage{unicode-math} % this also loads fontspec
  \defaultfontfeatures{Scale=MatchLowercase}
  \defaultfontfeatures[\rmfamily]{Ligatures=TeX,Scale=1}
\fi
\usepackage{lmodern}
\ifPDFTeX\else
  % xetex/luatex font selection
\fi
% Use upquote if available, for straight quotes in verbatim environments
\IfFileExists{upquote.sty}{\usepackage{upquote}}{}
\IfFileExists{microtype.sty}{% use microtype if available
  \usepackage[]{microtype}
  \UseMicrotypeSet[protrusion]{basicmath} % disable protrusion for tt fonts
}{}
\makeatletter
\@ifundefined{KOMAClassName}{% if non-KOMA class
  \IfFileExists{parskip.sty}{%
    \usepackage{parskip}
  }{% else
    \setlength{\parindent}{0pt}
    \setlength{\parskip}{6pt plus 2pt minus 1pt}}
}{% if KOMA class
  \KOMAoptions{parskip=half}}
\makeatother
\usepackage{graphicx}
\makeatletter
\newsavebox\pandoc@box
\newcommand*\pandocbounded[1]{% scales image to fit in text height/width
  \sbox\pandoc@box{#1}%
  \Gscale@div\@tempa{\textheight}{\dimexpr\ht\pandoc@box+\dp\pandoc@box\relax}%
  \Gscale@div\@tempb{\linewidth}{\wd\pandoc@box}%
  \ifdim\@tempb\p@<\@tempa\p@\let\@tempa\@tempb\fi% select the smaller of both
  \ifdim\@tempa\p@<\p@\scalebox{\@tempa}{\usebox\pandoc@box}%
  \else\usebox{\pandoc@box}%
  \fi%
}
% Set default figure placement to htbp
\def\fps@figure{htbp}
\makeatother
\setlength{\emergencystretch}{3em} % prevent overfull lines
\providecommand{\tightlist}{%
  \setlength{\itemsep}{0pt}\setlength{\parskip}{0pt}}
\usepackage{bookmark}
\IfFileExists{xurl.sty}{\usepackage{xurl}}{} % add URL line breaks if available
\urlstyle{same}
\hypersetup{
  hidelinks,
  pdfcreator={LaTeX via pandoc}}

\author{}
\date{}

\begin{document}

\section{\texorpdfstring{{Sistemi
Embedded}}{Sistemi Embedded}}\label{h.sv80xwggm41f}

{Sistemi (a microcontrollore) di elaborazione atti a svolgere una
specifica funzione o compito (special-purpose), non ri-programmabili e
incorporati in un sistema complesso. }

{}

{Le loro caratteristiche principali:}

{}

\begin{itemize}
\tightlist
\item
  {Funzionalità specifica}{: eseguono sempre le stesse cose
  ripetutamente;}
\item
  {Tightly constrained ed efficienza}{: memoria e CPU limitate;}
\item
  {Uso di design metrics}{: dimensioni, performance e consumo di
  energie;}
\item
  {Progettati per essere efficient}{i: efficienza energetica, di
  code-size, di run-time, di peso e di costo;}
\item
  {Affidabilità}{;}
\item
  {Reattivi e/o real time}{: permettono di avere un tempo di risposta
  basso o quasi nullo.}
\end{itemize}

\subsection{\texorpdfstring{{CPS - Cyber Physical
System}}{CPS - Cyber Physical System}}\label{h.xpwds9txibz5}

{Sono un'integrazione di diversa natura con lo scopo di controllare un
processo fisico e il suo adattamento in tempo reale attraverso
feedback.}

{}

\subsection{\texorpdfstring{{Architettura sistema
embedded}}{Architettura sistema embedded}}\label{h.1l1f95izhf97}

{Non c'è ne una di riferimento, ma esistono molteplici piattaforme, una
struttura di base potrebbe essere:}

{\pandocbounded{\includegraphics[keepaspectratio]{images/image14.png}}}

\subsubsection{\texorpdfstring{{CPU}}{CPU}}\label{h.hrpqdpbzkc1p}

{Possono essere 3:}

\begin{itemize}
\tightlist
\item
  {General-purpose}{:architettura e insieme di istruzioni predefinito,
  comportamento definito dal programma in esecuzione;}
\item
  {Application-specific processor (ASIP)}{: programmabili e ottimizzati
  per classi di applicazioni;}
\item
  {Single-purpose}{: progettati per implementare una specifica
  funzionalità o programma.}
\end{itemize}

{}

{Ma quale scegliere fra un processore }{GPP }{(general purpose) o
}{ASP}{~(application specific)?}

{I primi sono CPU di base che tramite l'utilizzo delle architetture di
Von Neumann e Harvard permettono l'uso di software personalizzato per
una specifica applicazione.}

{I secondi sono sistemi che danno soluzioni a specifiche applicazioni
con un numero limitato di funzioni che non giustificano architetture più
complesse.}

\subsubsection{\texorpdfstring{{MCU}}{MCU}}\label{h.qv2hj2bn8as8}

{Dispositivo integrato su un singolo circuito elettronico, evoluzione al
microprocessore ed utilizzato nei sistemi embedded.}

{Composto da un processore, una memoria permanente e volatile, canali di
I/O, gestore interrupt e blocchi specializzati.}

{Alcuni esempi possono essere gli Arduino dove si programma esternamente
e si passa il file già buildato pronto all'esecuzione.}

\subsubsection{\texorpdfstring{{SOC e Single-Board
PC}}{SOC e Single-Board PC}}\label{h.s2knihybh3bv}

{Sta per System on a chip, un unico chip incorpora un sistema completo.}

{}

{}

{}

{}

\section{\texorpdfstring{{Microcontrollori}}{Microcontrollori}}\label{h.62yon5dk42wx}

{Programmabile tramite sistema esterno, viene creato e buildato il file
poi passato al microcontrollore con USB o seriale.}

{I microcontrollori non hanno S.O. quindi tutto ciò passato viene
eseguito dal processore.}

{}

\subsection{\texorpdfstring{{Elementi
base}}{Elementi base}}\label{h.zeikue4uohn7}

{\pandocbounded{\includegraphics[keepaspectratio]{images/image38.png}}}

{}

\subsection{\texorpdfstring{{Esecuzione sul
controllore}}{Esecuzione sul controllore}}\label{h.n8dfw5ebtc0m}

{La CPU funziona tramite il ciclo }{fetch-decode-execute}{, nei
microcontrollori si usa l'architettura Harvard che al contrario di
quella Von Neumann memorizza istruzioni e dati in memorie separate
(codice e dati letti/scritti insieme).}

{}

{Il codice viene eseguito tramite }{super loop}{, che ha un
inizializzazione e un ciclo infinito che esegue un
}{task}{~ripetutamente. Questo approccio semplice la rende }{perfetta
per lo sviluppo}{~di applicazioni ma }{non permette una gestione della
temporizzazione}{~molto accurata.}

{}

{}

\subsection{\texorpdfstring{{GPIO (General Purpose
I/O)}}{GPIO (General Purpose I/O)}}\label{h.fkrozkclgg15}

{I microcontrollori normalmente dispongono di una serie di pin,
normalmente general purpose (programmabili per fare input o output).}

{I pin sono digitali (valore 0 o 1) o analogici (assumono qualsiasi
valore).}

\subsubsection{\texorpdfstring{{PIN}}{PIN}}\label{h.mcmjl1lpqu1d}

{Per i pin I/O occorre sempre prestare attenzione ai parametri di
funzionamento, cioè tensione (volt) da applicare o emessa in uscita e la
corrente (Ampere) che può ricevere o fornire.}

{Per evitare il floating (segnale input senza collegamento, cioè senza
tensione che li porta a captare qualsiasi disturbo) è possibile
equipaggiare i pin con circuiti pull-up.}

{}

{I}{~}{pin PWM}{~}{emulano in uscita un segnale analogico}{~a partire da
uno digitale, possibile grazie alla modulazione del duty cycle (\% di
tempo che il segnale è ad 1 rispetto a quella in cui è 0) di un'onda
quadra.}

{}

{I }{pin IRQ}{~r}{icevono segnali di interruzione}{~permettendo al
microcontrollore di r}{eagire ad eventi e eseguire istruzioni che non
sono }{direttamente }{nel loop}{~ma in routine separate.}

{All'arrivo di una richiesta si sospende l'esecuzione delle istruzioni,
si salva nello stack, restituisce il controllo alla routine di interrupt
(chiamata a interrupt handler o interrupt service routine).}

{Finito }{l'interrupt}{~si riprende dal punto di sospensione.}

{Si possono disabilitare le interruzioni, in questo modo siamo sicuri
che certi parti di codice verranno eseguite senza intoppi, le interrupt
request verranno eseguite quando le interruzioni saranno riabilitate (
se ci dimentichiamo di }{riabilitarle}{~il sistema potrebbe risentirne);
inoltre le interrupt possono interrompersi fra loro (interrupt
nesting).}

{}

{Se occorre elaborare un segnale analogico bisogna trasformarlo in
digitale tramite l'}{ADC}{~che mappa il valore continuo in un valore
discreto nel range 0-1023. Un segnale discreto può essere codificato.}

{}

{}

\subsection{\texorpdfstring{{Timers}}{Timers}}\label{h.y32cw7gzn0ff}

{Per realizzare un comportamento }{timer-oriented}{~i timer diventano
essenziali per misurare gli intervalli di tempo, gestire i PWM e
realizzare timeout.}

{Alcuni timer sono:}

\begin{itemize}
\tightlist
\item
  {Watch dog}{~timer}{: utilizzato per resettare il circuito in
  situazioni di blocco;}
\item
  {Power management:}{~permette di utilizzare il circuito con diversi
  livelli di risparmio energetico.}
\end{itemize}

{}

{}

\subsection{\texorpdfstring{{Protocolli di comunicazione e
bus}}{Protocolli di comunicazione e bus}}\label{h.fgbgjuderif6}

{Le interfacce di tipo seriale e parallelo servono per evitare lo
scambio di dati mediante singoli pin, le parallele inviano dati
parallelamente con l'utilizzo di diversi fili invece le seriali inviano
dati sequenzialmente.}

{}

\subsubsection{\texorpdfstring{{Seriali}}{Seriali}}\label{h.3l4skvdx7q2m}

{Ne esistono di due tipologie:}

\begin{itemize}
\tightlist
\item
  {Sincroni}{: c'è un clock, aumentano velocità e complessità
  (}\pandocbounded{\includegraphics[keepaspectratio]{images/image1.png}}{C
  e SPI);}
\item
  {Asincroni}{: non c'è clock,2 linee (trasmissione e ricezione) (USB,
  }{RS-485,}{~TTL); i parametri necessari per gestire la comunicazione
  basata su un protocollo prestabilito sono:}
\end{itemize}

\begin{itemize}
\tightlist
\item
  {baudRate}{:}{~velocità della trasmissione (bps), più è alta più
  possibili errori di trasmissione;}
\item
  {dataFrame}{:}{~struttura del pacchetto;}
\item
  {syncBit}{:}{~identifica eventuali bit di start e stop del pacchetto
  dati;}
\item
  {parityBit}{:}{~serve }{per}{~error checking (opzionale);}
\end{itemize}

{}

{Normalmente i bus seriali sono cablati con due linee separate con un
ricevitore (RX) e un trasmettitore (TX) collegati in maniera incrociata
tra due device:}

{\pandocbounded{\includegraphics[keepaspectratio]{images/image39.png}}}

{Se entrambi i dispositivi possono inviare e ricevere simultaneamente si
ha una comunicazione }{FULL-DUPLEX}{, se comunicano a turno
}{HALF-DUPLEX}{. Esiste anche la }{Serial Enable LCD}{, una
comunicazione seriale basata su singolo cavo.}

{}

{L'}{UART}{~ è il blocco circuitale in grado di convertire i dati da e
verso l'interfaccia seriale mettendola in comunicazione con
l'interfaccia parallela. È asincrono, si appoggia a dispositivi separati
per gestire i flussi di dati e utilizza un bus seriale.}

\paragraph{\texorpdfstring{{Seriali
sincroni}}{Seriali sincroni}}\label{h.ads57pe1ypfh}

{Abbinando un clock ad un'interfaccia seriale la si rende sincrona,
portando ad avere trasmissioni più rapide (ES: protocolli SPI e
}\pandocbounded{\includegraphics[keepaspectratio]{images/image1.png}}{C).}

{}

\pandocbounded{\includegraphics[keepaspectratio]{images/image1.png}}{C è
nato in Philips, veloce e robusto a due vie che utilizza
l'indirizzamento a 7 o 10 bit e si può arrivare a 3.4Mbit/s. Ha
un'architettura master-slave e più device condividono la SCL (Serial
Clock Line) e SDA (Serial Data Line).}

{}

{SPI è di Motorola ed è seriale full-duplex, basato su master-slave che
utilizza due linee MOSI (Master Out Slave In) per trasmettere e MISO
(Master In Slave Out) per ricevere; la linea Slave Select è riservata
per la selezione dello slave.}

{Grazie al Segnale di clock condiviso (SCLK) è sincrono.}

{}

{In sostanza:}

\begin{itemize}
\tightlist
\item
  {SPI: + veloce, + semplice, nessuna richiesta di pull-up;}
\item
  \pandocbounded{\includegraphics[keepaspectratio]{images/image1.png}}{C:
  solo 2 linee, indirizzamenti + semplici e aperti.}
\end{itemize}

{}

{}

{}

{}

\section{\texorpdfstring{{Sistemi SOC e
RTOS}}{Sistemi SOC e RTOS}}\label{h.fd7kc3qpt0kb}

{I SOC (System On a Chip) sono circuiti integrati in cui un unico chip
contiene un intero sistema operativo (CPU, Chipset, COntroller, Sistema
Video, I/O, Generatore di Clock e regolatori di tensione).}

{Nei SOC si installano S.O. detti RTOS (Real-Time Operating System) che
permettono la configurazione di sistemi reattivi e completi.}

{}

{}

\subsection{\texorpdfstring{{Sistemi
Operativi}}{Sistemi Operativi}}\label{h.heeqp9xmwn05}

{Nei S.O. moderni hardware e software comunicano attraverso moduli che
si occupano di diverse funzionalità (gestione di: processi, memoria,
rete, ecc), fornendo interfacce che permettono lo scambio di dati e
nascondendo dettagli implementativi tramite l'astrazione.}

{I S.O sono software di base con lo scopo di gestire hardware e software
ponendosi come mediatore tramite l'utilizzo di API chiamate }{chiamate
di sistema}{, eseguendo e gestendo i programmi e rendendo più facile ed
efficace l'utilizzo delle risorse attuando algoritmi di scheduling che
gestiscono la concorrenza dei programmi, garantendo sicurezza e
protezione.}

{}

{La vera struttura hardware viene nascosta }{al software}{~che vede solo
la macchina virtuale con una vista }{top-down}{~che aumenta la
portabilità dei programmi e semplificando la programmazione delle
applicazioni che potranno essere eseguite su macchine fisiche differenti
(se la macchina virtuale fornisce la stessa interfaccia).}

\subsubsection{\texorpdfstring{{Interfacce}}{Interfacce}}\label{h.24oczogbrynb}

{Ne esistono due principali:}

\begin{itemize}
\tightlist
\item
  {ISA}{: }{Istruction}{~Set Architecture, separa i livelli HW e SW,
  divisa in due parti: }{user ISA}{~che si occupa delle funzionalità
  visibili ai programmi utente e }{system ISA}{~che si occupa delle
  funzionalità del S.O;}
\item
  {ABI}{: Application Binary Interface permette l'accesso alle risorse
  HW e ai servizi del sistema ai programmi tramite }{user ISA}{~e
  }{System Call Interface}{~fornite dal S.O.}
\end{itemize}

{\pandocbounded{\includegraphics[keepaspectratio]{images/image37.png}}}

{}

{Per l'esecuzione dei programmi il S.O. può adottare due costrutti:}

\begin{itemize}
\tightlist
\item
  {Multiprogrammazione}{~quindi il }{caricamento in memoria centrale di
  più programmi e gestendo concorrentemente l\textquotesingle accesso
  all'utilizzo della CPU}{~}{quando i programmi utilizzano risorse
  differenti}{.}
\end{itemize}

{}

\begin{itemize}
\tightlist
\item
  {Multi-tasking}{~}{permette di eseguire programmi concorrentemente
  senza che questi debbano accedere ad altre risorse grazie }{le
  strategie}{~di schedulazione}{; se due task cercano di accedere alla
  stessa risorsa ho una dipendenza relativa ai dati.}
\end{itemize}

{}

{}

\subsection{\texorpdfstring{{RTOS}}{RTOS}}\label{h.e5swa6fzj5xj}

{S.O. real time creati apposta per i sistemi embedded che offrono
compattezza, estrema efficienza nelle gestione risorse e affidabilità. }

{Non offrono la multi-programmazione e l'esecuzione si limita ad una
sola applicazione ma permette l'esecuzione multi-task.}

{Questi S.O. devono reagire e rispondere entro determinati range
temporali prestabiliti e rispettare deadline, se le deadline sono sempre
rispettare si dicono }{Hard real-time}{~e si utilizzano nei sistemi
safety-critical. Se sono ammessi casi in cui le deadline possono non
essere rispettate si dicono }{Soft real-time}{~e si utilizzano
nell'elettronica di consumo.}

{}

{Nei sistemi real time }{tutto deve essere }{predicibile}{,}{~come il
tempo impiegato da un task, il tempo max per un'azione e la costanza
nella qualità dei cicli richiesti per un'operazione. }

{Le interruzioni sono comunque disponibili.}

{}

\subsubsection{\texorpdfstring{{Responsiveness e
Overhead}}{Responsiveness e Overhead}}\label{h.domkgs3y1lwg}

{Al contrario dei microcontrollori, che prediligono un loop con polling
continuo, per verifiche e esecuzioni di funzioni nei RTOS si sfruttano i
context switch che sono più }{predicibili}{~e non aggiungono overhead al
dispositivo con eventuale calo di prestazione.}

{}

\subsubsection{\texorpdfstring{{Gestione
risorse}}{Gestione risorse}}\label{h.2gcmwl2c1416}

{Per arbitrare e gestire l\textquotesingle accesso alle risorse vengono
messi a disposizione meccanismi centralizzati, come: }

\begin{itemize}
\tightlist
\item
  {Allocazione/deallocazione della memoria;}
\item
  {Semafori e mutex;}
\item
  {Meccanismi per gestire le sezioni critiche ei suoi problemi.}
\end{itemize}

{}

\subsubsection{\texorpdfstring{{Semplificazione}}{Semplificazione}}\label{h.1y3fkr23l5qv}

{Gli RTOS offrono servizi di astrazione per l'HW e l'utilizzo di
framework e piattaforme per gestire prodotti di terze parti; questo
approccio migliora il lavoro di team e un'agevolazione del debug con la
modulazione del software garantendo manutenibilità ed estensibilità}

{}

{I RTOS non sono congeniali per applicazioni semplici perchè basterebbe
un'architettura a looping o polling.}

{}

{}

{}

\section{\texorpdfstring{{Sensoristica}}{Sensoristica}}\label{h.6n3uemvvs09h}

{I }{sensori }{sono dispositivi trasduttori che misurano un certo
fenomeno fisico o concentrazione chimica fornendo scala o intervallo;
sono sia digitali che analogici.}

{In sostanza misurano tutto ciò che ``viene da fuori''.}

{Il suo scopo principale infatti è quello di misurare, cioè confrontare
due quantità omogenee, stabilendo in che rapporto una quantità incognita
stia rispetto ad un'altra che opera come riferimento. }

{Gli }{attuatori }{producono un effetto sull'ambiente (un led per
esempio) e sono sia analogici che digitali.}

{}

{Entrambi forniscono segnali analogici quindi continui ad risoluzione
infinita che rappresentano la grandezza fisica continua oppure
codificati in bit o logici (booleani) che rappresentano la grandezza
discreta.}

{Il segnale analogico per essere elaborato da un calcolatore deve essere
trasformato in digitale tramite un }{convertitore ADC}{, che campiona il
segnale (vengono presi dei campioni ad intervalli regolari di tempo) con
la }{quantizzazione}{~invece si }{approssimizza}{~il valore campionato
al più vicino digitale, può produrre un errore nella misurazione.}

{\pandocbounded{\includegraphics[keepaspectratio]{images/image20.png}}}

{}

{}

\subsection{\texorpdfstring{{Misure e
incertezze}}{Misure e incertezze}}\label{h.x95jeonshp9c}

{Ogni misura ha errori e/o incertezze, dove }{l'incertezza esprime la
dispersione dei valori che verificatasi in misurazioni ripetute si
trasforma in un errore}{.}

{In una misura il risultato è un numero ed una incertezza preceduta dal
simbolo }{±}{~e con un'unità di misura.}

{Un errore può essere: }

\begin{itemize}
\tightlist
\item
  {Sistematico: sempre stessa influenza;}
\item
  {Casuale:influenza che cambia in grandezza e segno;}
\item
  {Grossolani: riguardano lo strumento o l'operatore.}
\end{itemize}

{Se non vengono monitorate le condizioni ambientali possono sorgere
errori casuali, diversamente sono sistematici.}

{}

\subsection{\texorpdfstring{{Caratteristiche dei
sensori}}{Caratteristiche dei sensori}}\label{h.9v33fjl4mjfa}

{I sistemi di misura possono essere descritti tramite diverse
caratteristiche divisibili in statiche e dinamiche:}

\begin{itemize}
\tightlist
\item
  {Statiche: il tempo non influisce sullo strumento, definite dalla
  funzione
  }\pandocbounded{\includegraphics[keepaspectratio]{images/image2.png}}{,
  con X segnale in ingresso e Y in uscita.}
\end{itemize}

{Sono valutate in una situazione di normale funzionamento, dove la
caratteristica ideale è una retta con pendenza unitaria, ma nella realtà
il comportamento non è ideale (per via di imperfezioni costruttive) e
produce una deviazione di uscita rispetto al ``vero'', questa differenza
è l'errore del sensore.}

{\pandocbounded{\includegraphics[keepaspectratio]{images/image17.png}}}{\pandocbounded{\includegraphics[keepaspectratio]{images/image29.png}}}

{Per considerare l'errore bisogna definire }{la fascia di incertezza che
rappresenta la massima deviazione della sua retta di riferimento.}

{La scomposizione dell'errore nelle sue componenti serve per correggere
a posteriori la misura, altrimenti con la calibrazione o taratura
andiamo a }{mitigar}{e l'errore sul nascere.}

{}

{Isteresi}{~cioè la differenza massima tra i valori di uscita
corrispondente ad uno stesso ingresso, ottenuto prima per valori
crescenti, }{poi}{~decrescenti.}

{\pandocbounded{\includegraphics[keepaspectratio]{images/image16.png}}}

{Ripetibilità}{~è la capacità di riprodurre la stessa uscita con lo
stesso ingresso consecutivamente e nelle stesse condizioni.}

{\pandocbounded{\includegraphics[keepaspectratio]{images/image15.png}}}

{Linearità}{, lo scostamento della curva di taratura della retta di
riferimento, misurata agli estremi è detta terminale, a metà si dice
indipendente e ai minimi quadrati.}

{\pandocbounded{\includegraphics[keepaspectratio]{images/image36.png}}}

{}

{Risoluzione}{: l'ampiezza del passo delle uscite (distanza fra due
uscite consecutive) al variare dell'ingresso in tutto il suo range.}

{\pandocbounded{\includegraphics[keepaspectratio]{images/image28.png}}}

{}

{Sensibilità o guadagno}{~è la proporzionalità tra valore di ingresso e
valore di uscita, cioè la sensibilità in uscita di un sensore,
rappresenta la }{pendenza }{m}{~della}{~retta.}

{Offset}{~rappresenta il segnale in uscita anche in assenza del segnale
di ingresso, è il termine noto }{Q}{~della retta.}

{\pandocbounded{\includegraphics[keepaspectratio]{images/image6.png}}}

{Accuratezza}{~è la misura di quanto il valore letto dal sensore si
discordi dal valore corretto, la }{Precisione}{~descrive quanto un
sensore sia soggetto o meno ad errori accidentali (legata alla
ripetibilità).}

{\pandocbounded{\includegraphics[keepaspectratio]{images/image35.png}}}

{}

{}

\begin{itemize}
\tightlist
\item
  {Dinamiche: il tempo influisce ed è un fattore cruciale, alcune
  caratteristiche sono: }{tempo di risposta}{~cioè il tempo necessario
  per l'uscita di raggiungere una percentuale specifica del valore
  finale, }{tempo di salita}{~necessario affinché l'uscita vada da
  prefissato valore ad uno maggiore e }{costante di tempo}{~necessaria
  affinché l\textquotesingle uscita raggiunga il 63\% del valore
  finale.}
\end{itemize}

{}

{Il }{guadagno }{(K) del trasduttore è la }{costante di proporzionalità
fra i valori in ingresso e quelli di uscita prendendo in esame la
caratteristica statica}{~che idealmente deve essere lineare.}

{I trasduttori commerciali hanno una caratteristica statica reale che è
leggermente diversa da quella ideale, questo a causa di imperfezioni
costruttive. La qualità del sensore si misura in base a quanto la
caratteristica reale si discosta da quella ideale.}

{Per i trasduttori lineari la relazione tra la grandezza fisica misurata
e il segnale in uscita è una relazione lineare del tipo: Y = KX (dove K
è il guadagno).}

{}

{L'errore di linearità è la massima deviazione dell'uscita del
trasduttore rispetto alla caratteristica lineare che approssima al
meglio quella reale}{. La caratteristica lineare viene normalmente
ottenuta secondo il metodo dei minimi quadrati.}

{\pandocbounded{\includegraphics[keepaspectratio]{images/image18.png}}}

{}

\subsubsection{\texorpdfstring{{Errori
di\ldots{}}}{Errori di\ldots{}}}\label{h.lso0128fm55x}

{}

{Offset}{~o errore di fuori zero è il valore che assume l'uscita del
trasduttore quando la grandezza da misurare è nulla.}

{}

{Soglia}{~corrisponde al più basso livello di segnale rilevabile dal
sensore. Non è necessariamente coincidente con la grandezza da misurare.
}

{}

{Guadagno}{~descrizione del comportamento dello strumento, fatta in
relazione al tempo, ovvero si descrive il modo di funzionamento nel caso
in cui l'ingresso sia tempovariante (tempo di risposta).}

{}

{}

\subsection{\texorpdfstring{{Tipi di
sensori}}{Tipi di sensori}}\label{h.x5nq9sjvr6ue}

\subsubsection{\texorpdfstring{{Prossimità}}{Prossimità}}\label{h.6mmo5419jufc}

{Rileva la presenza di oggetti nelle vicinanze della parte sensibile del
sensore (induttiva, magnetica, ottica, ecc) nel raggio della sua portata
nominale (distanza entro la quale il sensore rileva oggetti).}

{}

{//La parte di sensori leggila nella slides 4 dalla diapositiva 22 a 46}

{}

{}

\subsection{\texorpdfstring{{Attuatori}}{Attuatori}}\label{h.af1fo8qh9k6x}

{Dispositivo che converte dell'energia da una forma ad un'altra,
l'interfacciamento avviene o con la corrente del GPIO o con corrente
maggiore.}

\subsubsection{\texorpdfstring{{TIpi}}{TIpi}}\label{h.eceqh6vx1wyv}

{LED}{: ~dispositivi optoelettronici che sfruttano alcuni materiali
semiconduttori in grado di produrre fotoni attraverso un fenomeno di
emissione spontanea; questo li rende altamente efficienti e
particolarmente affidabili.}

{}

{Display LCD}{: ~pannelli formati da due strati di vetro tra i quali è
racchiuso un liquido e numerosi contatti elettrici in grado di applicare
un campo elettrico al liquido, comandando una piccola porzione di
liquido che identifica un pixel.}

{}

{Motori elettrici:}{~all'interno uno statore ed un rotore che,
producendo un campo magnetico, generano il movimento del rotore. Vi sono
motori di diverso tipo, i più comuni nei sistemi embedded sono: }

\begin{itemize}
\tightlist
\item
  {Corrente continua}{: rotazione continua ad elevata velocità o coppia,
  controllo della velocità di rotazione. ~La corrente attraversa }{un
  avvolgimento}{~}{nel}{~rotore che genera il campo magnetico;}
\item
  {Passo-Passo (Stepper)}{: motore brushless che suddivide la rotazione
  in molti step, la posizione può quindi essere controllata
  accuratamente ed hanno un'escursione di 360 gradi;}
\item
  {Servo motori}{: come per il precedente è possibile pilotare l'angolo
  del rotore ma la posizione è assoluta e non relativa alla posizione
  corrente, escursione di 180 gradi.}
\end{itemize}

{}

\subsubsection{\texorpdfstring{{Pilotaggio di circuiti
esterni}}{Pilotaggio di circuiti esterni}}\label{h.br8d4e9bpojs}

{Il Relè permette di aprire/chiudere il secondo circuito mediante
l'azione di un }{elettro-magnete}{~pilotato dal primo circuito.}{~}

{Possono essere:}

\begin{itemize}
\tightlist
\item
  {~}{NC}{~(normalmente chiusi);}
\item
  {~}{NO}{~(normalmente aperti) se senza tensioni in ingresso sulla
  bobina il contatto in uscita risulta ``spento'', }
\end{itemize}

{Pilotati con diverse tensioni in ingresso e possono gestire tensioni in
uscita.}

{}

{I fotoaccoppiatori invece si usano principalmente per trasferire un
segnale da }{un apparato}{~ad un altro tenendoli isolati elettricamente.
Un led interno si attiva, pilotato da un circuito a bassa tensione, e
attiva una fotocellula, chiudendo l\textquotesingle interruttore del
secondo circuito.}

\subsubsection{\texorpdfstring{{Partitore di
tensione}}{Partitore di tensione}}\label{h.6w4lwn4qah2t}

{Permette di produrre in uscita una tensione voluta a partire da una
tensione in ingresso più alta}

{}

{Viene usato per evitare problemi sui livelli di tensione differenti che
sensori e circuiti hanno, in modo da evitare danni.}

\subsubsection{\texorpdfstring{{Resistenze di
pull-up/down}}{Resistenze di pull-up/down}}\label{h.k07th63a7hgj}

{Sono molto comuni per sopperire all\textquotesingle alta impedenza dei
pin dei device che li può portare ad uno stato di «Floating ». }

{Si inserisce quindi una piccola resistenza o tra il pin e Vdd
(resistenza di Pull-Up) o tra il pin e Vss (GND o V0) (resistenza di
Pull-Down).}

{}

{}

\section{\texorpdfstring{{Architetture
software}}{Architetture software}}\label{h.sn0or1jfjn6r}

{}

{La progettazione di un sistema embedded è differente da quella dei
sistemi general purpose, infatti vi sono tre aspetti fondamentali
fortemente connessi fra loro:}

\begin{itemize}
\tightlist
\item
  {Architettura;}
\item
  {Applicazione {[}requisiti funzionali e non{]};}
\item
  {Metodologie di sviluppo.}
\end{itemize}

{}

{Questo tipo di sviluppo è molto complesso e per raggiungere un buon
prodotto occorrono analisi, simulazioni alternate con l'uso di prototipi
dove la simulazione non fornisce valori accettabili e strumenti avanzati
per il testing dell'hardware.}

{Inoltre si consiglia l'utilizzo di una metodologia ibrida, in grado di
integrare metodologie di sviluppo software e hardware.}

{}

\subsection{\texorpdfstring{{Metodologie}}{Metodologie}}\label{h.u73zrd3zdg1h}

\subsubsection{\texorpdfstring{{Co-design}}{Co-design}}\label{h.p5rhj2cyrynh}

{~Utilizzata quando sono presenti componenti hardware e software
insieme, mira a ottimizzare il processo di progetto andando ad aumentare
la produttività e accorciando i tempi di sviluppo con il riuso dei
componenti.}

{}

{\pandocbounded{\includegraphics[keepaspectratio]{images/image25.png}}}

{}

{Co-Specifica}{: si analizza e si fa una specifica insieme;}

{Co-Partizionamento}{: divisione componenti tra hardware e software, un
buon compromesso mitiga costi e ottimizza le prestazioni;}

{Co-Sintesi}{: sviluppo componenti, diviso in 4 fasi:}

\begin{itemize}
\tightlist
\item
  {comunicazione}{: interfacciamento tra HW e SW;}
\item
  {personalizzazione delle specifiche}{: ~metodi di comunicazione per
  permettere l'interfacciamento;}
\item
  {specifiche HW;}
\item
  {specifiche SW}{;}
\end{itemize}

{Co-Scheduling}{: assegnamento timing di esecuzione per ridurre la
latenza al minimo;}

{Co-Simulazione}{: verifica funzionalità hardware e software;}

{Co-validation}{: verifica del sistema e diversi livelli di astrazione.}

{}

\subsubsection{\texorpdfstring{{Modello a
cascata}}{Modello a cascata}}\label{h.vbllb6kz6jyd}

{Modello classico e vecchio stile, dove ogni fase genera un output
necessario per quella successiva. Gestione semplice per progetti piccoli
ma pericoloso per progetti più complicati e lunghi rendendo impossibile
effettuare modifiche ai livelli superiori.}

\subsubsection{\texorpdfstring{{Modello a
spirale}}{Modello a spirale}}\label{h.or73hjlu0eeu}

{In realtà è un metamodello creato da Barry Boehm e rappresenta diversi
cicli di vita, si basa su quattro fasi: }{pianificazione, analisi,
sviluppo e verifica}{.}

{Il raggio delle fasi è il costo accumulato}{~(+ è alto + il costo
aumenta) e la dimensione angolare il progresso, favorisce uno sviluppo
costante e definito.}

\subsubsection{\texorpdfstring{{Metodologia agili
}{{[}agiail{]}}}{Metodologia agili {[}agiail{]}}}\label{h.o8zp6d4mhvxb}

{Usata dalla Vem, al centro c'è il cliente che si incontra ogni due
settimane per }{mostrare}{~i progressi fatti e verificare che siano in
linea con quello che il cliente vuole.}

{Basato sul metodo }{``Scram'',}{~ma con regole meno ferree, dove si
divide il progetto in sprint (massimo 3 settimane) dove si lavora senza
}{sottoteam}{~e ogni membro del progetto sceglie un task da sviluppare
(scelto nello sprint planning); ogni task è creato dal capo progetto e
sono molto base.}

{A fine sprint si mostra al cliente le nuove funzionalità nello sprint
review; successivamente si ricomincia con lo sprint planning dove creo e
distribuisco i task nuovi e decido quali task vecchi non completati
portare allo sprint successivo.}

{}

\subsection{\texorpdfstring{{Modellazione}}{Modellazione}}\label{h.rep2q9kk4y3q}

{Uno dei momenti più importanti durante lo sviluppo di un software
poiché fondamentale per rappresentare il sistema definendo soluzioni più
astratte offrendo migliore comprensione, estensibilità, portabilità e
modularità.}

{Attraverso dei modelli ben definiti e basati su paradigmi di
programmazione è possibile concettualizzare il sistema
}{rappresentandone gli}{~aspetti salienti ed astraendo i meno
significativi.}

{}

{La modellazione su MCU avviene tramite un controllore che contiene la
logica di controllo e gestisce le risorse, è la parte normalmente attiva
con un modello a loop di controllo.}

{Gli elementi controllati invece modellano le risorse gestite dal
controllore per svolgere i }{compiti, contengono}{~le funzionalità utili
al controllore e sono tipicamente passivi, solo la parte riusabile.}

{}

\subsection{\texorpdfstring{{Paradigma
OO}}{Paradigma OO}}\label{h.o5iwkbhd1bvq}

{Paradigma ad oggetti offre un buon livello di astrazione con proprietà
come modularità, incapsulamento e meccanismi per il riuso e
l\textquotesingle estensibilità.}

{Inoltre permette di avere un supporto naturale alla modellazione
software di oggetti reali. }

{}

{Il modello a loop ci sono alcuni limiti, infatti i loop continuano ad
essere eseguiti anche quando non vi sono cambiamenti di input
(efficienza) e se il ciclo è piuttosto complesso a livello
computazionale si possono introdurre dei ritardi, perdendo eventuali
variazioni che vengono compiute durante l'elaborazione (reattività).}

{}

{Il controller del modello oo non è una classe ma un }{agente}{,
ovvero:}

{Entità attiva dotata di un flusso di controllo logico autonomo.
progettato per svolgere uno o + task che richiedono di elaborare
informazioni provenienti da input dei sensori e di agire su attuatori in
output, comunicando con altri agenti}{.}

{}

{}

\subsection{\texorpdfstring{{Macchine a stati
finiti}}{Macchine a stati finiti}}\label{h.cj56ac1cxmgo}

{Modello di sistema a dinamica discreta, dove }{ogni input viene mappato
in un output}{, a seconda del suo stato corrente ad ogni reazione
rilevata, con un numero finito di stati possibili.}

{}

{Stato di sistema}{: condizione in cui si trova in un certo istante
temporale, ovvero l'insieme di tutte le attività passate, che
determinano la reazione del sistema agli input futuri.}

{}

{Reazione}{: step che il sistema effettua, istantanee e scatenate
dall\textquotesingle ambiente o da eventi.}

{}

{\pandocbounded{\includegraphics[keepaspectratio]{images/image27.png}}}

{Le macchine a stati finiti possono essere:}

\begin{itemize}
\tightlist
\item
  {Asincrone}{: cioè event triggered, dove la reazione (o transizione)
  avviene a fronte di un evento in input;}
\end{itemize}

{}

\begin{itemize}
\tightlist
\item
  {Sincrone:}{~cioè time triggered, dove le reazioni avvengono ad
  intervalli regolari di tempo, per questo motivo questo tipo di
  macchine sono adatte alla gestione di sistemi time-oriented.}
\end{itemize}

{La lettura di un input con un certo periodo di tempo è detto
}{campionamento}{, se si sceglie bene il periodo di campionamento è
possibile non perdere eventi e mantenere la reattività al costo di
risorse vista la rapidità del campionamento; il }{MEST }{(Minimum Event
Separation TIme) è l'intervallo più piccolo che può esserci fra due
eventi di input quindi }{scegliend}{o un periodo più piccolo del MEST
rileveremo tutti gli eventi.}

{La }{latenza}{~è il tempo che intercorre tra l'evento di input e la
generazione dell'output.}

{}

\subsection{\texorpdfstring{{Task}}{Task}}\label{h.9smmjrc0r4nj}

\subsubsection{\texorpdfstring{{Organizzazione task
concorrenti}}{Organizzazione task concorrenti}}\label{h.1x7xymhf3eif}

{L'architettura a task concorrenti nasce dalla necessità di modellare e
progettare applicativi articolati dove è fondamentale decomporre e
modularizzare le funzionalità.}

{Ogni task ha un compito, un'unità di lavoro e il suo comportamento è
descritto da una macchina a stati; dove il comportamento di tutti i task
è l'insieme delle varie macchine a stati.}

{}

{I task possono essere suddivisi in sotto-task in maniera ricorsiva
aumentando la modularità portando ad un sistema chiaro, manutenibile,
estendibile e riusabile.}

{}

{Bisogna ricordarsi che i task sono fra loro concorrenti e le loro
dipendenze vanno gestite mediante forme di coordinazione cooperative o
competitive.}

{}

{La metodologia cooperativa usa scheduling di tipo round-robin che è più
semplice ed è privo di corse critiche, a scapito di un loop infinito in
uno dei task che può portare gli altri in una situazione di starvation.}

\subsubsection{\texorpdfstring{{Dipendenze
task}}{Dipendenze task}}\label{h.30liklh7gu2m}

{Non riuscendo a garantire un\textquotesingle indipendenza completa
esistono forme di dipendenza:}

\begin{itemize}
\tightlist
\item
  {Temporale: per far partire un altro task bisogna aspettare che
  l'altro finisca;}
\item
  {Producer/Consumer: un task deve aspettare l\textquotesingle output di
  un altro;}
\item
  {Relative ai dati: c'è una risorsa in comune, bisogna aspettare che
  l'altro task la liberi.}
\end{itemize}

{}

{Nelle macchine a stati l'interleaving delle azioni non avviene.}

\subsubsection{\texorpdfstring{{Comunicazione
task}}{Comunicazione task}}\label{h.dv7ctx9nnl2c}

{Il caso più semplice è utilizzare una variabile globale, così si ha una
comunicazione sincrona. C'è la possibilità di gestire i messaggi in
maniera asincrona utilizzando per ogni task una coda.}

{}

\subsection{\texorpdfstring{{Scheduling e
prestazioni}}{Scheduling e prestazioni}}\label{h.ncqlbrd22hng}

{Per ogni task è necessario valutare la loro durata per evitare
malfunzionamenti, si possono definire eccezioni di overrun quando il
tempo di esecuzione delle azioni oltrepassa il periodo pre stabilito
andando a ``rubare'' tempo di esecuzione al task successivo (timer
overrun nel caso di scheduler interrupt-driven).}

{}

{È possibile definire se si verificherà un overrun analizzando il codice
per stimare il Worst-Case Execution Time (con task che usano periodi
diversi bisogna considerare il WCET con degli iper-periodi).}

\pandocbounded{\includegraphics[keepaspectratio]{images/image3.png}}

{}

{Ottenendo un valore \textgreater100\% si verificherà un overrun.}

{}

{Jitter}{: ritardo che intercorre dal momento che un task è pronto per
essere eseguito e ~il momento in cui viene effettivamente eseguito,
dando priorità ai task con periodo piccolo }{minimizziamo}{~il jitter
medio.}

{Deadline}{: intervallo di tempo entro il quale un task deve essere
eseguito dopo essere diventato ready.}

\subsubsection{\texorpdfstring{{Priorità}}{Priorità}}\label{h.3wn8nniqfz8a}

{Per priorità si intende l'ordinamento con cui lo scheduler esegue i
task. Normalmente vi possono essere due tipologie di scheduler: }

{}

\begin{itemize}
\tightlist
\item
  {Priorità statica: ogni task ha un livello di priorità che non cambia
  durante l'esecuzione, si usa la policy shortest-deadline first a meno
  che la deadline non sia uguale altrimenti si passa alla
  shortest-period first;}
\item
  {Priorità dinamica: la priorità è determinata in esecuzione e la si da
  a chi ha una deadline più vicina alla fine.}
\end{itemize}

\subsubsection{\texorpdfstring{{Preemptive e
cooperativi}}{Preemptive e cooperativi}}\label{h.4s6063abo0da}

{Sono due tipi di scheduler, il primo può togliere il processo ad un
task in esecuzione prima che abbia completato; il secondo aspetta finché
il task non sia stato portato a termine ed è utilizzato per realizzare
le macchine a stati finiti.}

{}

\subsection{\texorpdfstring{{Architetture su
eventi}}{Architetture su eventi}}\label{h.423v6i9vviot}

{A basso livello le interruzioni di input rappresentano eventi che
gestiscono meglio le risorse ed evitano polling.}

{Si possono utilizzare gli eventi anche per realizzare architetture ad
eventi di alto livello come pattern-observer o macchine a stati finiti
asincrone.}

\subsubsection{\texorpdfstring{{Pattern-observer}}{Pattern-observer}}\label{h.upb7fgx2ksss}

{Si sfruttano le interruzioni per far si che all'occorrenza di
un'interruzione si chiamino i listener.}

{\pandocbounded{\includegraphics[keepaspectratio]{images/image22.png}}}

\subsubsection{\texorpdfstring{{Macchine a stati finiti
asincrone}}{Macchine a stati finiti asincrone}}\label{h.ph7i4vadxl2x}

{Sfruttano le interruzioni per realizzarle, la valutazione delle
reazioni avviene a seguito di un evento, non esiste il periodo anche se
l'evento può essere di tipo temporale.}

{}

{}

\section{\texorpdfstring{{IOT}}{IOT}}\label{h.nc5w5fgcmwdu}

{Coniato da Kevin Ashton nel `99, automatizzare l'inserimento dei dati
inerenti al mondo fisico in Internet con sensori con l'ipotesi di un
futuro interconnesso.}

{}

{Definizione standard:}

{``Things having identities and virtuale personalities operating in
smart spaces using intelligent interfaces to connect and communicate
within social, }{environmental}{, and user contexts''}

{}

{resto nelle slides}

{}

{}

\section{\texorpdfstring{{Comunicazione}}{Comunicazione}}\label{h.f0q9ldwnyrlb}

{In ottica IoT la connessione alla rete può essere:}

\begin{itemize}
\tightlist
\item
  {Diretta: si ha un modulo connessione;}
\item
  {Indiretta: usando un gateway, usando un sistema terzo.}
\end{itemize}

{}

\subsection{\texorpdfstring{{Wireless}}{Wireless}}\label{h.8t2yaufy3cbt}

{Da non confondere con il WiFi perchè il wireless comprende il WiFi e
tutto ciò che non usa fili per la comunicazione, in sostanza la
radiocomunicazione.}

{Sfruttando canali radio possiamo trasmettere dati attraverso segnali
elettromagnetici che appartengono alla frequenza radio dello spettro
elettromagnetico; }{tali comunicazioni sono satellitari o terrestri.}

{}

\subsection{\texorpdfstring{{Sistema
radiocomunicazione}}{Sistema radiocomunicazione}}\label{h.rbznmriabvm7}

{Composto da:}

\begin{itemize}
\tightlist
\item
  {Canale trasmissivo: tutto ciò che serve per trasmettere;}
\item
  {Modem: codifica da bit-segnale a elettrico e viceversa;}
\item
  {Codec: codifica e decodifica digitalmente un segnale.}
\end{itemize}

{}

{La rete può essere:}

{\pandocbounded{\includegraphics[keepaspectratio]{images/image24.png}}}{\pandocbounded{\includegraphics[keepaspectratio]{images/image23.png}}}{\pandocbounded{\includegraphics[keepaspectratio]{images/image32.png}}}

\subsubsection{\texorpdfstring{{WiFi}}{WiFi}}\label{h.a2b7682fwljn}

{Standard IEEE }{802,11x,}{~supportato da IP, TCP, UDP.}

{Data rate elevato con un range di }{150m,}{~frequenza 2,4 o 5 GHz e con
prestazioni non favorevoli per i sistemi embedded.}

\subsubsection{\texorpdfstring{{Reti
cellulari}}{Reti cellulari}}\label{h.r7efevy3e79i}

{Standard evoluti dal GPRS a LTE, data rate che può essere alto o basso
in base allo standard, consumi abbastanza alti e range dell'ordine dei
chilometri.}

{Successivamente è nato il Narrow Band IoT che utilizza le reti
cellulari ma con copertura altissima e consumi ridotti a scapito del
data rate.}

\subsubsection{\texorpdfstring{{Sistemi
RF}}{Sistemi RF}}\label{h.wzjt7hjt8fxp}

{Interfacce via radio con data rate a 1 Mbps, range elevati, frequenza
di 2,4 GHz e senza supporto di protocolli ad alto livello.}

\subsubsection{\texorpdfstring{{Bluetooth}}{Bluetooth}}\label{h.fy1dhv4tojr8}

{Protocollo 802.15.1, sfrutta la scoperta dinamica di dispositivi entro
10 metri di raggio, costi di produzione e energia bassi, frequenze
libere e con }{il frequency hopping}{~commuta la banda e }{permette di
ottimizzare l'utilizzo della frequenza}{.}

{La rete del BT è detta piconet basata sull'architettura master-slave
dove ogni dispositivo comunica con 7 slave e questi slave comunicano in
maniera sincronizzata (grazie al clock master) uno per volta con il
master. Più }{piconet}{~connesse insieme sono dette scatternet, dove gli
slave possono appartenere a più reti ma il master può essere slave solo
di una sola piconet.}

\subsubsection{\texorpdfstring{{ZigBee}}{ZigBee}}\label{h.8f9s7w3xf46t}

{Il protocollo IEEE 802.15.4, si basa su uno standard di comunicazione e
ne vuole definire uno complementare al WiFi (quindi personale e a basso
costo/velocità), il ZigBee usa questa specifica con comunicazioni entro
10 metri, transfer rate di 250 }{kbits}{~e utilizzo frequenze
868/915/2450 MHz.}

{La rete è composta da FFD (coordinatore che parla con tutti) e RFD
(parla solo con il FED).}

{Il Zigbee è ideato per l'IoT con basso consumo e basso costo, i
dispositivi si dividono in:}

\begin{itemize}
\tightlist
\item
  {ZigBee Coordinator (ZC): è un ponte fra reti, solo uno per rete e
  tiene le chiavi di sicurezza {[}FFD{]};}
\item
  {ZigBee Router (ZR): trasmettono dati fra dispositivi {[}FFD{]};}
\item
  {ZigBee End Device (ZED): parla solo con coordinatore e router
  {[}RFD{]}}
\end{itemize}

\subsubsection{\texorpdfstring{{LoraWAN}}{LoraWAN}}\label{h.w2pf0p6s6dp4}

{Poca potenza ma grossa area di copertura, costo energetico basso e
utilizzo frequenze radio.}

{Struttura con topologia a stella avente un gateway per connettersi ad
internet che comunica con un network server (di solito in cloud) che
gestisce i downlink e gli end device possono connettersi a più gateway.}

{Usando bande libere il protocollo ha dei duty-cycle, cioè un periodo di
tempo per usare la banda, in sostanza il data rate è limitato}{. }

{}

{Se aumenta il data rate ci sarà più velocità di trasmissione a scapito
dello spreading factor}{~(portata delle trasmissioni). }

{Con spreading factor + alti occorrono pacchetti più piccoli e consumi
più alti.}

{}

{La comunicazione }{LoraWAN}{~è bidirezionale con classi di device
diverse:}

\begin{itemize}
\tightlist
\item
  {Classe A: il device di campo inizia sempre a comunicare e solo dopo
  si può rispondere entro una finestra temporale;}
\item
  {Classe B: come nella classe precedente ma si aprono anche finestre di
  ricezione schedulate;}
\item
  {Classe C: finestra sempre aperta a meno che non ci sia una
  trasmissione in corso.}
\end{itemize}

{Bisogna sempre }{ricevere}{~un uplink prima di inviare un downlink.}

\subsubsection{\texorpdfstring{{RFID}}{RFID}}\label{h.uor8twwgml7f}

{Tracking a distanza tramite tag che hanno un id univoco, i tag possono
essere: passivi (non alimentato), attivo (ha batterie e comunica in
broadcast), BAP (invia informazioni solo se sollecitato).}

\subsubsection{\texorpdfstring{{NFC}}{NFC}}\label{h.51tzaem78ovm}

{Fornisce connettività radio bidirezionale a corto raggio, quando si
accostano }{i'initiator}{~e il target viene creata una rete
peer-to-peer.}

\subsubsection{\texorpdfstring{{Beacon}}{Beacon}}\label{h.d26yzvr1y8xp}

{Basata sul BLE}{~utile per localizzazione indoor, si usa un device
detto beacon a basso consumo che trasmette di continuo il suo ID.}

{}

\subsection{\texorpdfstring{{Middleware}}{Middleware}}\label{h.get83xlysmi6}

{Struttura di riferimento per la standardizzazione della connettività
via Internet dei device IoT.}

{Il loro scopo è fornire interoperabilità (}{capacità di un sistema o di
un prodotto informatico di cooperare e di scambiare informazioni}{)
tramite supporto a protocolli di alto livello, discovery dinamico,
aggiornamenti, disconnettere dispositivi problematici, fornire supporto
per la memorizzazione, scalabilità e sicurezza.}

{}

{Si hanno diversi livelli:}

{\pandocbounded{\includegraphics[keepaspectratio]{images/image33.png}}}

{Dove:}

\begin{itemize}
\tightlist
\item
  {Event processing}{: rileva gli eventi e li processa;}
\item
  {Aggregation}{: ~aggrega e gestisce le comunicazioni e fa da ponte fra
  i vari protocolli;}
\item
  {Comunication}{:}{~diversi protocolli possono essere utilizzati, http,
  MQTT, CoAP, XMPP;}
\item
  {Devices}{: Dispositivi di vario tipo con o senza connessione diretta,
  identificati a livello HW.}
\end{itemize}

{}

\subsection{\texorpdfstring{{MQTT}}{MQTT}}\label{h.bwwyt3uel7zg}

{Protocollo a scambio di messaggi }{publish-subscribe}{, basato sul
modello broker.}

{Vantaggioso per il suo basso overhead, possibilità di funzionare su
diversi protocolli (TCP, ZigBee) e con anche la presenza di firewall e
garantisce la bontà dei messaggi tramite QOS.}

{}

{Il QOS ha tre livelli:}

\begin{enumerate}
\tightlist
\item
  {At most once}{: viene inviato il messaggio, non importa chi e se
  qualcuno lo legge, per questo detto }{fire and forget}{;}
\item
  {At least once}{: c'è la sicurezza che arrivi a destinazione almeno
  una volta;}
\item
  {Exactly once}{: il messaggio arriva una sola volta.}
\end{enumerate}

{}

\subsection{\texorpdfstring{{CoAP}}{CoAP}}\label{h.kl8ly2no8m04}

{Basato su HTTP con approccio client-server utilizzando UDP.}

{\pandocbounded{\includegraphics[keepaspectratio]{images/image4.png}}}

\subsection{\texorpdfstring{{XMPP}}{XMPP}}\label{h.5dblsdiaksrv}

{Basato su scambio di messaggi con struttura message-brokers. Supporta
anche altri metodi come il point-to-point, il request-response e
}{l\textquotesingle asyncronous}{~message, costruendo federazioni di
server XMPP possiamo ottimizzare la scalabilità.}

{}

\subsection{\texorpdfstring{{Modelli di
comunicazione}}{Modelli di comunicazione}}\label{h.kfqc3jt943rc}

\subsubsection{\texorpdfstring{{A scambio di
messaggi}}{A scambio di messaggi}}\label{h.7cvvt0vso49r}

{Comunicazione tramite invio e ricezione di messaggi utilizzando send o
receive.}

{}

{Possono essere:}

\begin{itemize}
\tightlist
\item
  {Diretta}{: comunicazione diretta fra processi utilizzando un
  identificativo univoco;}
\end{itemize}

{o}

\begin{itemize}
\tightlist
\item
  {Indiretta}{: sfruttando canali di comunicazione.}
\end{itemize}

{}

{}

\begin{itemize}
\tightlist
\item
  {Sincrona}{: la send va a buon fine quando il messaggio è stato
  recepito tramite receive;}
\end{itemize}

{o}

\begin{itemize}
\tightlist
\item
  {Asincrona}{: la }{send}{~va a buon fine quando il messaggio viene
  inviato.}
\end{itemize}

{}

{}

\begin{itemize}
\tightlist
\item
  {Descrizione dei messaggi.}
\end{itemize}

{}

{Non bisogna sottovalutare la bufferizzazione dei messaggi, che nel caso
asincrono è fondamentale. Nel caso il buffer si riempia bisogna gestire
il comportamento della send scegliendo fra i seguenti meccanismi:}

{}

\begin{itemize}
\tightlist
\item
  {La send fallisce}{~con scarto del messaggio;}
\item
  {La send si blocca}{~finchè non c'è disponibilità nel buffer;}
\item
  {La }{send}{~ha successo}{~e si riscrive il messaggio nel buffer più
  nuovo o vecchio.}
\end{itemize}

{}

{Per la rappresentazione del messaggio si usa un linguaggio condiviso da
tutti come Json o XML e adottare ontologie (sintassi e strutture
comuni).}

{}

\subsubsection{\texorpdfstring{{Publish-subscribe}}{Publish-subscribe}}\label{h.itdwrktdfgmo}

{\pandocbounded{\includegraphics[keepaspectratio]{images/image31.png}}}

{Utilizzato nelle strutture ad eventi e composto da:}

\begin{itemize}
\tightlist
\item
  {Topics}{: canali in cui è possibile inviare dei messaggi e che è
  possibile «osservare»;}
\item
  {Publisher}{: possono inviare messaggi sui topics;}
\item
  {Subscriber}{: si registrano ai topics per ricevere tutti i messaggi
  pubblicati su questi.}
\end{itemize}

{}

\section{\texorpdfstring{{Domande
esame}}{Domande esame}}\label{h.87cdfifn1rau}

{\pandocbounded{\includegraphics[keepaspectratio]{images/image10.png}}}{\pandocbounded{\includegraphics[keepaspectratio]{images/image5.png}}}{\pandocbounded{\includegraphics[keepaspectratio]{images/image9.png}}}{\pandocbounded{\includegraphics[keepaspectratio]{images/image26.png}}}

{Nella domanda 30 la risposta giusta è: serve un server NTP (e non PTN)
per questo sono tutte sbagliate !!!}

{\pandocbounded{\includegraphics[keepaspectratio]{images/image34.png}}}{\pandocbounded{\includegraphics[keepaspectratio]{images/image30.png}}}{\pandocbounded{\includegraphics[keepaspectratio]{images/image7.png}}}{\pandocbounded{\includegraphics[keepaspectratio]{images/image8.png}}}{\pandocbounded{\includegraphics[keepaspectratio]{images/image11.png}}}{\pandocbounded{\includegraphics[keepaspectratio]{images/image19.png}}}{\pandocbounded{\includegraphics[keepaspectratio]{images/image12.png}}}{\pandocbounded{\includegraphics[keepaspectratio]{images/image13.png}}}{\pandocbounded{\includegraphics[keepaspectratio]{images/image21.png}}}

\subsection{\texorpdfstring{{Domande sbagliate per
orale}}{Domande sbagliate per orale}}\label{h.90lvnwsebbwr}

\begin{enumerate}
\tightlist
\item
  {Cosa viene eseguito }{al riavvio}{~di un device Arduino? Un
  bootloader.}
\item
  {Nelle organizzazioni di task concorrenti\ldots{} ogni task è
  descritto tramite una singola macchina a stati finiti.}
\item
  {Le porte che permettono la gestione dei pin I/O\ldots{} sono gestiti
  da uno o più registri }{special}{~purpose.}
\item
  {Nei dispositivi IoT non aventi un clock interno ma connessi a
  internet, cosa posso utilizzare per ottenere l'orario corretto? Si usa
  un server NTP che però è esterno e non dentro il router, se fosse
  dentro il router anche senza internet avremmo avuto l'orario
  corretto.}
\item
  {Quale fra queste NON è un'attività dei S.O? Gestione della
  comunicazione tra programmi.}
\item
  {Nel protocollo }{LoraWAN}{~un messaggio di downlink può sempre essere
  inviato in quale di queste casistiche? In nessuna, i messaggi downlink
  vengono inviati solo come risposta di uno uplink non importa la
  classe.}
\item
  {Il duty cicle nel protocollo }{LoraWan\ldots{}}{~NON è inversamente
  proporzionale alla dimensione dei pacchetti, ma è il massimo tempo di
  trasmissione per un determinato periodo di tempo.}
\item
  {Quando una variabile condivisa da task deve essere in sezione critica
  per garantire il corretto funzionamento? Solo se i task sono in
  scrittura, se è in lettura invece non c'è bisogno.}
\item
  {Se ho un sensore di temperatura collegato alla corrente elettrica, in
  grado di comunicare ad intervalli di 30 minuti }{con un requisito
  funzionale di una temperatura media ogni 5 minuti (ogni 5 minuti
  voglio il valore medio di temperatura di quel periodo)}{~campionamento
  ogni 500ms quale tempo utilizzo per ogni step della macchina a stati?
  Un secondo.}
\item
  {Quale tra le seguenti caratteristiche è legate a errori accidentali?
  La precisione.}
\item
  {Per gestire la comunicazione in modo asincrono\ldots{} si possono
  utilizzare code di messaggi.}
\item
  {Quali delle seguenti affermazioni su UART è falsa?
  }{L'UART}{~utilizza una comunicazione sincrona. Infatti è asincrono,
  si appoggia a dispositivi separati per gestire i flussi di dati e
  utilizza un bus seriale.}
\item
  {I gateway }{LoraWAN\ldots{}}{~sono l'unico accesso alla rete
  internet.}
\end{enumerate}

\end{document}
