\section{Domande esame}\label{domande-esame}

\begin{itemize}
\item
  \emph{Principali standard alla base del web}:
\end{itemize}

\begin{quote}
I principali standard sono: \textbf{HyperTextMarkupLanguage} (formato
per le pagine), \textbf{UniformResourceIdentifier} (identificatore
globale delle risorse) e \textbf{HyperTextTrasferProtocol} (protocollo
per scambio di messaggi tramite TCP);
\end{quote}

\begin{itemize}
\item
  \emph{Cosa si intende per accessibilità}:
\end{itemize}

Secondo l'articolo 2:

\begin{quote}
"\textbf{Accessibilita}\textquotesingle": la capacità dei sistemi
informatici, nelle forme e nei limiti consentiti dalle conoscenze
tecnologiche, di erogare servizi e fornire informazioni fruibili, senza
discriminazioni, anche da parte di coloro che a causa di disabilità
necessitano di tecnologie assistive o configurazioni particolari.
\end{quote}

\begin{itemize}
\item
  \emph{Quali sono i principi alla base delle WCAG 2.0?}
\end{itemize}

\begin{quote}
Il \textbf{principi base} sono 4 e sono:
\end{quote}

\begin{itemize}
\item
  \emph{Percepibile;}
\item
  \emph{Utilizzabile;}
\item
  \emph{Comprensibile;}
\item
  \emph{Robusto}.
\end{itemize}

Da questi principi discendono 12 linee guida che forniscono indicazioni
per

prendere il contenuto più accessibile possibile; e per ogni linea guida
vengono

forniti dei criteri di successo che appartengono a 3 livelli (A, AA,
AAA) dove

\begin{quote}
ogni livello soddisfa i propri criteri, eventualmente quelli del
precedente oppure viene fornita una versione alternativa conforme al
proprio livello.
\end{quote}

\begin{itemize}
\item
  \emph{Cosa si intende per tecnologia assistiva?}
\end{itemize}

\begin{quote}
Sempre secondo l'articolo 2:

"\textbf{Tecnologie assistive}": gli strumenti e le soluzioni tecniche,
hardware e software, che permettono alla persona disabile, superando o
riducendo le condizioni di svantaggio, di accedere alle informazioni e
ai servizi erogati dai sistemi informatici.
\end{quote}

\begin{itemize}
\item
  \emph{Definire il concetto di «cascata» nei fogli di stile:}
\end{itemize}

\begin{quote}
Nel caso esistano più regole CSS in conflitto fra loro si usa il
concetto di cascata, dove le dichiarazioni vengono ordinate in base ai
seguenti fattori:
\end{quote}

\begin{itemize}
\item
  Media;
\item
  Importanza di dichiarazione;
\item
  Origine della dichiarazione
\item
  Specificità del selettore;
\item
  Ordine delle dichiarazioni
\end{itemize}

\begin{itemize}
\item
  \emph{Qual è la differenza tra usabilità e user experience?}
\end{itemize}

\begin{quote}
L'\textbf{usabilità} è il grado in cui un prodotto può essere usato da
particolari utenti per raggiungere certi obiettivi con efficacia,
efficienza e soddisfazione in uno specifico contesto
d\textquotesingle uso.

L'\textbf{user experience} descrive la reazione
dell\textquotesingle utente di fronte all\textquotesingle interazione
con lo strumento in base a \emph{dimensione pragmatica} (usabilità del
sistema), \emph{dimensione estetica/edonistica} (piacevolezza estetica)
e \emph{dimensione simbolica} (forza del brand e identificazione).

In parte i due concetti si sovrappongono: \textbf{l'usabilità}
(funzionale e semplice) fa riferimento alla capacità di svolgere un
compito con efficienza, efficacia e soddisfazione, mentre
l'\textbf{esperienza d'uso} (soggettiva e dinamica) include anche gli
aspetti legati alla sfera delle emozioni; \emph{usabilità come una
superstrada (funzionale e semplice) e UX come strada di montagna (più
complessa ma emozionante).}
\end{quote}

\begin{itemize}
\item
  \emph{Qual è l'architettura alla base del Web?}
\end{itemize}

\begin{quote}
Il web si basa su una architettura \textbf{client-server}, dove il
client inizia \emph{sempre} la comunicazione (\ul{PULL}) richiedendo i
contenuti e il server può solo rispondere (\ul{PUSH}) anche se in certi
casi è possibile prediligere l'immediata disponibilità di dati.

Una tecnica ibrida è la cosiddetta \ul{PSEUDO-PUSH} dove il client ad
intervalli regolari e in maniera autonoma richiede (\ul{POOL}) se sono
disponibili nuovi contenuti;
\end{quote}

\begin{itemize}
\item
  \emph{Cosa significa separare struttura/contenuto e presentazione
  nell'ambito delle pagine Web?}
\end{itemize}

\begin{quote}
Significa dividere la parte HTML, utile per definire il contenuto della
pagina, dalla parte CSS cioè la parte che definisce come il contenuto
verrà presentato.
\end{quote}

\begin{itemize}
\item
  \emph{Quali sono le principali caratteristiche di Javascript?}
\end{itemize}

\begin{quote}
\textbf{JavaScript} è un linguaggio di scripting orientato agli oggetti
(debolmente) e agli eventi, è un linguaggio interpretato, debolmente
tipizzato e sia front che end site;
\end{quote}

\begin{itemize}
\item
  \emph{Quali sono i principali tipi di variabili in Javascript?}
\end{itemize}

\begin{quote}
Le variabili vengono dichiarate con \textbf{var} e non hanno nessun
tipo, la dichiarazione senza il var (Esempio: \emph{pluto = 18}) è detta
implicita e introduce sempre variabili globali; con la parola chiave var
è detta esplicita.

Con una nuova versione sono state inserite le parole chiave \textbf{let}
(scope di blocco) e \textbf{const} (queste variabili non possono essere
riassegnate).
\end{quote}

\begin{itemize}
\item
  Che cos'è un ipertesto?
\end{itemize}

\begin{quote}
Insieme non lineare di documenti collegati l\textquotesingle uno
all\textquotesingle altro per mezzo di connessioni logiche e rimandi (
link ) che consentono all\textquotesingle utente di costruirsi di volta
in volta un autonomo percorso di lettura.
\end{quote}

\begin{itemize}
\item
  Che cos'è il markup?
\end{itemize}

\begin{quote}
Deriva dal contesto della tipografia, dove si usavano annotazioni per
marcare parti di testo da correggere;
\end{quote}

\begin{itemize}
\item
  Che cos'è un linguaggio di markup?
\end{itemize}

Il \textbf{linguaggio di markup} è un linguaggio con una specifica
sintassi che permette di annotare un documento fornendone una
interpretazione delle sue

parti.

Possono essere o procedurali (dove il markup specifica le istruzioni per

visualizzare le parti di testo) oppure descrittive (dove indicano il
ruolo di ogni

tag) come HTML;

\begin{itemize}
\item
  Che cos'è un browser?
\end{itemize}

\begin{quote}
Il \textbf{browser} è un visualizzatore di documenti ipertestuali e
multimediali in HTML, si possono usare \emph{plug-in} per visualizzare
correttamente tutti i formati e permette di visualizzare applicazioni
web anche senza il server (dopo averle scaricate);
\end{quote}

\begin{itemize}
\item
  Che cos'è HTML?
\end{itemize}

\begin{quote}
L'\textbf{HTML} è un linguaggio di markup progettato per dare formato a
documenti scientifici con struttura ipertestuale. Il contenuto è
composto da \emph{tag} (elementi di markup) che definiscono la
struttura;
\end{quote}

\begin{itemize}
\item
  Cosa sono i fogli di stile?
\end{itemize}

\begin{quote}
Il \textbf{CascadingStyleSheets} è un linguaggio specifico per definire
gli stili e gli aspetti presentazionali di pagine HTML;
\end{quote}

\begin{itemize}
\item
  Che cos'è VUE.js?
\end{itemize}

\begin{quote}
Un framework progressivo che utilizza il modello MVVM
(Model-View-ViewModel), composto quindi da modello (implementazione del
dominio dati gestito dall'applicazione), vista (componente grafico
renderizzato all'utente) e modello per la vista (collante tra i
precedenti componenti).
\end{quote}

\section{Web}\label{web}

Il World Wide Web è uno spazio dove i documenti e altre risorse sono
identificate tramite URL, uniti tramite ipertesti e accessibili da
Internet.

\subsection{Tecnologie}\label{tecnologie}

Ancora oggi le tre tecnologie fondamentali sono:

\begin{itemize}
\item
  \textbf{HyperTextMarkupLanguage} \textbf{:} un linguaggio di markup
  progettato per dare formato a documenti scientifici con struttura
  ipertestuale. Il contenuto è composto da \emph{tag} (elementi di
  markup) che definiscono la struttura;
\item
  \textbf{UniformResourceIdentifier} \textbf{:} identificatore globale
  per identificare le risorse nel web, a sua volta si divide in
  \emph{URL} e \emph{URN};
\item
  \textbf{HyperTextTrasferProtocol :} protocollo per la comunicazione,
  necessario per scambiarsi i messaggi su una rete informatica.
\end{itemize}

\subsubsection{URI}\label{uri}

Gli URI sono una sintassi usata in WWW per definire i nomi e gli
indirizzi di risorse su Internet, vengono definiti come:

\begin{itemize}
\item
  \emph{URL:} una sintassi che contiene informazioni immediatamente
  utilizzabili per accedere alla risorsa;
\item
  \emph{URN:} una sintassi che permetta una etichettatura permanente e
  non ripudiabile della risorsa.
\end{itemize}

La rappresentazione di un URI è la seguente:

\emph{\textbf{schema :} \textbf{{[}// authority{]} path {[}? query{]}
{[}\# fragment{]}}}

Esempio: http://www.sito.com/dir1/dir2/pluto.html

La parte di \emph{query} e \emph{fragment} individua delle risorse
secondarie, andando ulteriormente nello specifico.

Un URI può essere assoluto, lo schema è completo, oppure
\emph{gerarchico o relativo} (detto \emph{URI reference}) dove solo una
parte del URI assoluto è riportato

\subsubsection{HTML}\label{html}

Ultimo standard HTML 5, il documento è definito dal \emph{DOCTYPE},
incluso fra i tag \emph{\textless html\textgreater{}} e strutturato in
\emph{\textless head\textgreater{}} e
\emph{\textless body\textgreater{}}.

In HTML esistono elementi di \textbf{metadatazione}, che descrivono il
documento specificando caratteristiche, comportamento, ecc.

Vanno inclusi nell'head e sono:

\begin{itemize}
\item
  \emph{\textless title\textgreater{}} -\textgreater{} titolo del
  documento {[}MAX: 1{]};
\item
  \emph{\textless base\textgreater{}} -\textgreater{} indica il path
  base {[}MAX: 1{]};
\item
  \emph{\textless link\textgreater{}} -\textgreater{} crea relazioni tra
  i documenti e risorse;
\item
  \emph{\textless meta\textgreater{}} -\textgreater{} aggiungono altri
  metadati utili al browser;
\item
  \emph{\textless style\textgreater{}} -\textgreater{} permette di
  includere stili all'interno del documento;
\end{itemize}

\subsubsection{CSS}\label{css}

Il \textbf{CascadingStyleSheets} è un linguaggio specifico per definire
gli stili e gli aspetti presentazionali di pagine HTML.

Ha lo scopo fondamentale di separare contenuto e presentazione nelle
pagine Web, l'HTML serve per definire il contenuto, il CSS serve per
definire come il contenuto deve essere presentato.

Come si capisce dal nome il CSS prevede la presenza di fogli di stile
multipli, che agiscono uno dopo l\textquotesingle altro in cascata, nel
caso esistano più regole CSS in conflitto fra loro, dove le
dichiarazioni vengono ordinate in base ai seguenti fattori:

\begin{itemize}
\item
  Media;
\item
  Importanza di dichiarazione;
\item
  Origine della dichiarazione
\item
  Specificità del selettore;
\item
  Ordine delle dichiarazioni
\end{itemize}

Il CSS si divide in 4 livelli:

\begin{enumerate}
\def\labelenumi{\arabic{enumi}.}
\item
  \textbf{CSS level 1:} specifica caratteristiche tipografiche e di
  presentazione per gli elementi di un documento HTML \emph{{[}NON
  SUPPORTATO COMPLETAMENTE{]}};
\item
  \textbf{CSS level 2/2.1:} introduce il supporto per media multipli e
  un layout più sofisticato \emph{{[}NON SUPPORTATO COMPLETAMENTE{]}};
\item
  \textbf{CSS level 3:} alcune sezioni sono già recommendation (Colors,
  Selectors, Fonts) ma non ancora finita;
\item
  \textbf{CSS level 4:} in discussione.
\end{enumerate}

Il CSS si può inserire nella propria pagina attraverso l'attributo
\emph{\textless style\textgreater{}} dentro il tag di riferimento
(\textbf{inline}), posizionato dentro \emph{l'header}
(\textbf{interno}), importato dall\textquotesingle{}\emph{header}
(\textbf{esterno importato}) oppure indicando un foglio esterno
\emph{.css} dentro il tag \emph{link} (\textbf{esterno}).

\paragraph{Selettori}\label{selettori}

Dentro al CSS abbiamo le seguenti tipologie di selettori:

\begin{itemize}
\item
  \textbf{Universale:} \emph{(*)} match con tutti gli elementi;
\item
  \textbf{Di tipo:} \emph{(E)} match con gli elementi E
\item
  \textbf{Di prossimità:} \emph{(E F, E\textgreater F, E+F,
  E\textasciitilde F)} match con elementi F che siano discendenti, figli
  diretti, immediatamente seguenti o fratelli di E
\item
  \textbf{Di attributi:} \emph{(E{[}foo{]}, E{[}foo="bar"{]},
  E{[}foo\textasciitilde="bar"{]}, E{[}foo\^{}="bar"{]})} match con gli
  elementi E che possiedono l\textquotesingle attributo specificato o
  che ha un valore particolare
\item
  \textbf{Di classe:} \emph{(E.bar E\#bar)} il primo si usa solo per le
  classi, il secondo identifica gli elementi il cui attributo di tipo id
  vale "bar"
\item
  \textbf{Di pseudo-classi:} \emph{(E:link, E:visited, E:active,
  E:hover, E:focus, E:enabled, E:checked, E:lang(c))}
\end{itemize}

La specificità di un selettore è data da una quadrupla xywz dove:

\begin{itemize}
\item
  \textbf{x:} 1 se la dichiarazione è nell'attributo style, 0
  altrimenti;
\item
  \textbf{y:} numero di id specificati nel selettore;
\item
  \textbf{w:} numero di classi, attributi e pseudo-classi specificati
  nel selettore;
\item
  \textbf{z:} numero di elementi e di pseudo-elementi specificati nel
  selettore.
\end{itemize}

\subsection{Architettura}\label{architettura}

Il web si basa su una architettura \textbf{client-server}, dove il
client inizia \emph{sempre} la comunicazione (\ul{PULL}) richiedendo i
contenuti e il server può solo rispondere (\ul{PUSH}) anche se in certi
casi è possibile prediligere l'immediata disponibilità di dati.

Una tecnica ibrida è la cosiddetta \ul{PSEUDO-PUSH} dove il client ad
intervalli regolari e in maniera autonoma richiede (\ul{POOL}) se sono
disponibili nuovi contenuti.

\subsubsection{Client}\label{client}

Il client è il browser, un visualizzatore di documenti ipertestuali e
multimediali in HTML, si possono usare \emph{plug-in} per visualizzare
correttamente tutti i formati e permette di visualizzare applicazioni
web anche senza il server (dopo averle scaricate)

\subsubsection{Server}\label{server}

Il server è un'applicazione che risponde alle richieste di risorse
locali (individuate da un id. univoco) da parte dei client, infatti non
può iniziare l'interazione con il client.

\subsubsection{Cookies}\label{cookies}

Sono blocchi di dati non interpretabili lasciati dai server per salvare
informazioni utili al ristabilimento dei diritti di una risorsa;
sostanzialmente contengono dati utili alla gestione delle sessioni.

\subsection{Programmazione}\label{programmazione}

Esistono due tipi di applicazioni web, \textbf{lato client} e
\textbf{lato server}.

\subsubsection{\texorpdfstring{JavaScript {[} lato client {]}
}{JavaScript {[} lato client {]} }}\label{javascript-lato-client}

Linguaggio di scripting orientato agli oggetti (debolmente, meglio
definirlo \emph{object-based}) e agli eventi, è un linguaggio
interpretato, debolmente tipizzato e sia front che end site.

In questo linguaggio, come negli altri \emph{prototype-based}, non
esistono le classi ma i prototipi, istanze astratte e sempre
modificabili e accessibili.

Il paradigma che utilizza è detto \emph{event-driver}.

Alcuni oggetti principali sono la \emph{window} (top-level con le
proprietà e i metodi della finestra principale), il \emph{navigator}
(oggetto con le proprietà del client), \emph{location} ('URL del
documento corrente), \emph{history} (l\textquotesingle array degli URL
acceduti durante la navigazione) e il \emph{document} (rappresenta il il
DOM).

\subsubsection{DOM}\label{dom}

Il \textbf{Document Object Model} è la rappresentazione dei documenti
strutturati come modello orientato agli oggetti, ogni documento caricato
dal browser genera un DOM che specifica sotto forma di gerarchia di
oggetti, tutti gli elementi di quel documento.

Il DOM ha vari oggetti:

\begin{itemize}
\item
  \emph{DOMDocument}: il documento di cui si sta parlando;
\item
  \emph{DOMElement}: ogni singolo elemento del documento;
\item
  \emph{DOMAttr}: ogni singolo attributo del documento;
\item
  \emph{DOMText}: ogni singolo nodo di testo del documento;
\item
  \emph{DOMNode:} specifica i metodi per accedere a tutti gli elementi
  di un nodo di un documento.
\end{itemize}

\subsubsection{PHP}\label{php}

Linguaggio di scripting, interpretato, originariamente concepito per la
programmazione di pagine web dinamiche lato server.

\subsubsection{AJAX}\label{ajax}

Insieme di tecnologie (\emph{\textbf{A}synchronous \textbf{J}avaScript
\textbf{a}nd \textbf{X}ML}) utilizzate per la realizzazione di
\textbf{RIA}, cioè applicazioni web client/server-side interattive.

Si basa sullo scambio di dati in background con conseguente
aggiornamento dinamico della pagina.

XML può essere sostituito anche con JSON (\emph{AJAJ}).

\subsubsection{Applet}\label{applet}

Applicazioni scritte in Java incapsulate nella pagina web (tag:
\emph{\textless object\textgreater{}}), oramai obsolete.

\subsubsection{Flash}\label{flash}

Altra tecnologia deprecata che permetteva l'utilizzo di oggetti
dinamici, per funzionare il browser doveva avere alcuni plug-in.

\subsubsection{Vue.js}\label{vue.js}

Nato nel 2014, uno dei framework più facili da utilizzare, è di tipo
progressivo quindi vanno installate librerie aggiuntive per avere tutto.

Permette di sviluppare interfacce web reattive che sfruttano il
\textbf{dual-binding}.

Usa il modello \emph{MVVM}
(\textbf{M}odel-\textbf{V}iew-\textbf{V}iew\textbf{M}odel), composto
quindi da \ul{modello} (implementazione del dominio dati gestito
dall'applicazione), \ul{vista} (componente grafico renderizzato
all'utente) e \ul{modello per la vista} (collante tra i precedenti
componenti).

Al contrario del \emph{controller}, che è una porzione di codice che
esegue particolari logiche di business e ritorna una View, il
\emph{ViewModel} rappresenta un modello parallelo al Model, e che viene
direttamente bindato alla View.

\emph{Il Controller esegue logiche di business prima del rendering della
View, il ViewModel definisce il comportamento dell'applicazione a
runtime.}

\subsection{Framework}\label{framework}

Sono librerie che rendono più ricco, semplice e sofisticato l'uso di una
tecnologia.

I \textbf{full-stack framework} forniscono piattaforme completamente
integrate.

\subsubsection{Web solution stack}\label{web-solution-stack}

Insieme di componenti o sottosistemi software che sono necessari per
creare una piattaforma completa in modo che nessun software aggiuntivo
sia indispensabile allo sviluppo di applicazioni; uno stack è costituito
da 4 elementi:

\begin{enumerate}
\def\labelenumi{\arabic{enumi}.}
\item
  \textbf{Sistema operativo;}
\item
  \textbf{Database};
\item
  \textbf{Web server};
\item
  \textbf{Linguaggio di programmazione}.
\end{enumerate}

Alcuni esempi possono essere:

\begin{itemize}
\item
  \emph{LAMP} (\textbf{L}inux, \textbf{A}pache, \textbf{M}ySQL e
  \textbf{P}HP),
\item
  \emph{WAMP} (\textbf{W}indows, \textbf{A}pache, \textbf{M}ySQL e
  \textbf{P}HP),
\item
  \emph{WISA} (\textbf{W}indows, \textbf{I}IS, \textbf{S}QL Server e
  \textbf{A}SP.NET) è un esempio di architettura solo di prodotti
  Microsoft,
\item
  \emph{MEAN} (\textbf{M}ongo, \textbf{E}xpress, \textbf{A}ngular e
  \textbf{N}ode.js) dove non c'è un S.O. di riferimento ma componenti
  per lo sviluppo client.
\end{itemize}

\section{Guerre dei Browser}\label{guerre-dei-browser}

Il primo browser fu \textbf{Mosaic}, prototipo creato al CERN e poi reso
pubblico, successivamente venne \textbf{Netscape Navigator} che ottenne
un successo immediato.

\subsection{Prima {[}1994 - 1998{]}}\label{prima-1994---1998}

Dopo la falsa partenza con \textbf{Microsoft Network} Microsoft decise
di realizzare \textbf{Internet Explorer}, sia quest'ultimo sia Netscape
cambiano e migliorano HTML, Opera rimane nell'ombra.

Alla fine Microsoft ha la meglio includendo il proprio browser in
Windows 95, nel `98 Netscape ammette la perdita e con il rilascio del
codice sorgente nascerà Mozilla Firefox (2002).

\subsection{Seconda {[}2004 - 2016/22{]}}\label{seconda-2004---201622}

Nel 2004 iniziarono ad affermarsi browser nuovi e gratis come
\textbf{Chrome} (più usato dal 2016) o \textbf{Safari} che spodestano
completamente \textbf{Explorer}, ritirato completamente a giugno 2022.

\subsection{Dopoguerre}\label{dopoguerre}

La prima guerra dei browser portò una perdita di adesione allo standard
HTML, per questo Berners-Lee (creatore HTML e il web) e Cailliau
fondarono il \textbf{W3C} (\textbf{W}orld \textbf{W}ide \textbf{W}eb
\textbf{C}onsortium) un'istituzione che dirige lo sviluppo degli
standard come HTML 4.

Nel 2004 Opera e Firefox, dopo la bocciatura da parte del \textbf{W3C},
fondarono il \textbf{Web Hypertext Application Technology} (WHAT WG) che
cambiò alcuni aspetti di competenza del W3C; nel 2007 W3C riaprì il
working group e con i membri del WHAT crearono HTML 5.

\emph{HTML è considerato un living standard, cioè non si stabilizza
essendo in continua evoluzione !!!}

\section{Web Design}\label{web-design}

Nel processo di pianificazione e creazione di un sito web, vengono
introdotti alcuni elementi come:

\subsection{Testo}\label{testo}

Il media con rappresentazione digitale più naturale, necessita di un
sistema di codifica e ogni simbolo può avere più stili.

\subsubsection{Gilfo e font}\label{gilfo-e-font}

Il \textbf{glifo} è una entità tipografica che realizza la
rappresentazione visiva della forma del carattere, un insieme di glifi è
detto \textbf{font}.

Un font è un insieme di glifi caratterizzati da un certo stile grafico o
progettati per svolgere una data funzione.

Sono classificabili tramite le seguenti caratteristiche:

\begin{itemize}
\item
  \textbf{proporzionale/monospace}:
\end{itemize}

\begin{enumerate}
\def\labelenumi{\arabic{enumi}.}
\item
  lunghezza variabile dei glifi = proporzionali.
\item
  stessa lunghezza dei glifi = monospace.
\end{enumerate}

\begin{itemize}
\item
  \textbf{serif/sans-serif}:
\end{itemize}

\begin{enumerate}
\def\labelenumi{\arabic{enumi}.}
\item
  glifi senza grazie: sans serif.
\item
  glifi con grazie: serif.
\end{enumerate}

\emph{Le \textbf{grazie} sono allungamenti ortogonali}

\subsection{Colori}\label{colori}

Indicati tramite il nome inglese o il codice \textbf{rgb}.

I colori RGB (colori primari
\emph{\textbf{R}ed-\textbf{G}reen-\textbf{B}lue}) si esprimono in
esadecimale utilizzando la sintesi additiva. Le terne di colori vanno da
0 a 255.

\subsubsection{Color Wheel}\label{color-wheel}

Si utilizza per scegliere uno schema di colore più armonico e
accessibile possibile, è formata da 12 colori così suddivisi:

\begin{itemize}
\item
  3 primari: rosso, verde, blu;
\item
  3 secondari: ciano, magenta, giallo;
\item
  6 terziari.
\end{itemize}

I colori sono analoghi (vicini) o complementari (opposti).

\subsection{Layout pagina}\label{layout-pagina}

Il layout dominante è il \textbf{mobile first} quindi dove si basa il
sito per una visualizzazione orizzontale, essendo più complesso conviene
farlo subito, è un'applicazione di tipo \textbf{progressive enhancement}
dove si parte dalla condizione più vincolante e si creano design
progressivamente più ricchi ( si supera il \textbf{graceful degradation}
usato per i desktop).

\subsection{Usabilità}\label{usabilituxe0}

L'\textbf{usabilità} trova applicazione nell'informatica nel settore
dell'\emph{ergonomia cognitiva} cioè il modo in cui l'utente costruisce
aspettative sul funzionamento di un prodotto, per questo il modello di
chi ha progettato (\emph{design model}) e il modello del funzionamento
(\emph{user model}) devono coincidere il più possibile.

\emph{In poche parole l'usabilità è: ``il grado in cui un prodotto può
essere usato da particolari utenti per raggiungere certi obiettivi con
efficacia, efficienza e soddisfazione in uno specifico contesto
d\textquotesingle uso''}.

\subsection{UX}\label{ux}

L'\textbf{user experience} descrive la reazione
dell\textquotesingle utente di fronte all\textquotesingle interazione
con lo strumento in base a \emph{dimensione pragmatica} (usabilità del
sistema), \emph{dimensione estetica/edonistica} (piacevolezza estetica)
e \emph{dimensione simbolica} (forza del brand e identificazione).

\emph{In poche parole l'UX sono: le percezioni e le reazioni di un
utente che derivano dall'uso o dall'aspettativa d'uso di un prodotto,
sistema o servizio.}

\subsubsection{Personas e Scenarios}\label{personas-e-scenarios}

Le personas sono descrizioni degli utenti rappresentativi mentre gli
scenarios descrivono in modo realistico la sequenza di azioni che una
persona compie utilizzando un servizio.

\subsubsection{Focus group}\label{focus-group}

Sono discussioni riguardo al prodotto fatte insieme a membri dell'utenza
target, si usano per esplorare i vantaggi e gli svantaggi di un set
limitato e predefinito di proposte/opzioni presentate in modo
strutturato ad un gruppo di persone.

\section{Accessibilità}\label{accessibilituxe0}

Iniziamo con la definizione di \emph{\textbf{accessibilità}}:

\emph{La capacità dei sistemi informatici, nelle forme e nei limiti
consentiti dalle conoscenze tecnologiche, di erogare servizi e fornire
informazioni fruibili, senza discriminazioni, anche da parte di coloro
che a causa di disabilità necessitano di tecnologie assistive o
configurazioni particolari;}

e di \emph{\textbf{tecnologie assistive}}:

\emph{gli strumenti e le soluzioni tecniche, hardware e software, che
permettono alla persona disabile, superando o riducendo le condizioni di
svantaggio, di accedere alle informazioni e ai servizi erogati dai
sistemi informatici.}

Una tecnologia accessibile non beneficia solo un utente con disabilità
(\ul{visive}, \ul{uditive} e \ul{motorie}) ma anche coloro che
utilizzano tecnologie differenti; il \textbf{curbcut effect} è
esattamente questo cioè quando anche altre tipologie di utenti hanno
benefici da una tecnologia accessibile.

Un esempio di tecnologia assistiva è lo \emph{screen reader},
un\textquotesingle applicazione software, utilizzata prevalentemente
dalle persone non vedenti, che identifica ed interpreta il testo
mostrato sullo schermo di un computer, presentandolo tramite sintesi
vocale o attraverso un display braille.

\subsection{WCAG}\label{wcag}

Sono le linee guida individuate dal \textbf{WAI} (\textbf{W}eb
\textbf{A}ccessibility \textbf{I}nitiative), un gruppo di lavoro
sull'accessibilità del Web nell'ambito del \emph{W3C}.

Il \textbf{principi base} del WCAG 2.0 sono:

\begin{itemize}
\item
  \emph{Percepibile;}
\item
  \emph{Utilizzabile;}
\item
  \emph{Comprensibile;}
\item
  \emph{Robusto}.
\end{itemize}

Da questi principi discendono 12 linee guida che forniscono indicazioni
per

prendere il contenuto più accessibile possibile; e per ogni linea guida
vengono

forniti dei criteri di successo che appartengono a 3 livelli (A, AA,
AAA) dove

ogni livello soddisfa i propri criteri, eventualmente quelli del
precedente oppure viene fornita una versione alternativa conforme al
proprio livello.

\subsection{Daltonismo}\label{daltonismo}

Indica la cecità ai colori, ovvero l\textquotesingle inabilità a
percepire i colori, può essere totale (acromasia o acromatopsia) oppure
parziale (discromatopsia o dicromatismo).

Con la tecnica \textbf{G183} si intende di usare un contrasto pari al
3:1 con i colori vicino ai testi e aggiungere altri aiuti visivi con
focus sui link e sui controlli che utilizzano i colori per essere
identificati.

La tecnica è legata anche alla definizione di \emph{relative luminance}.
